\documentclass{report}
\usepackage[margin=2cm]{geometry}

\title{ECE 106 - Physics of Electrical Engineering 2}
\author{Andy Zhang}
\date{Winter 2014}

\begin{document}
\maketitle
\tableofcontents
\chapter{Electric Fields}
	\section{Properties of an Electric Charge}
		\begin{itemize}
			\item Two types of charge: \textbf{Positive} and \textbf{Negative}
			\item Charges of the same sign repel one another and charges with opposite signs attract one another.
			\item Electric charge is always conserved.
		\end{itemize}
	\section{Charging Objects by Induction and Conduction}
		\subsection{Insulators and Conductors}
		\begin{itemize}
			\item Conductors are a material where electrons not bound to atoms and are free to move through the material.
			\item Insulators are materials in which all electrons are bound to atoms and cannot move freely through the material.
		\end{itemize}
		\subsection{Charging by Induction}
			Using a charged object, charges in a conductor can be repulsed out of an object through ground. The objects do not have to touch for this to happen.
		\subsection{Charging by Conduction}
			Rub two objects so that charges on one rub off to the other.
	\section{Coulomb's Law}
		To generalize the properties of the electric force, we find that the magnitude of the electric force between two point charges is given by Coulomb's Law:\\
		\centerline{$F_e = k_e \frac{|q_1||q_2|}{r^2}$}\\
		where $k_e$ is known as Coulomb's constant and has the value:\\
		\centerline{$k_e = 8.9876 \times 10^9 N \cdot m^2/C^2 = \frac{1}{4\pi \epsilon_0}$ }\\
		Note that the force is a vector and with multiple charges, the force is equal to the sum of the vector forces. As mentioned previously, the direction depends on the two charges.
	\section{Particle in an Electric Field}
		An electric field is said to exist in the region of space around a charged object, the source charge. We define the electric field vector $\vec{E}$ at a point to be:\\
		\centerline{$\vec{E} \equiv \frac{\vec{F}_e}{q_0} = k_e \frac{q}{r^2}\hat{r}$}\\
		Like the force, an electric field at a certain point is the sum of all electric fields due to multiple point charges.
	\section{Electric Field of a Continuous Charge Distribution}
		When calculating the electric field at a point due to a continuous distribution of charge such as a surface, we can divide the charge distribution into many small charge $\Delta q$.\\
		We then take the sum of all of these charges:\\
		\centerline{$\vec{E} = k_e \int_{}^{} \frac{dq}{r^2} \hat{r}$} \\
		Popular examples would include a uniform ring of charge where many of the forces would cancel out due to symmetry and you'd calculate the formula with a $\cos \theta$ to only get 1 component. This approach would be the same for a disk of uniform charge.
	\section{Electric Field Lines}
		The electric field vector $\vec{E}$ is tangent to the electric field line at each point. The denser the number of electric field lines, the stronger the field and vice versa.\\
		For \textbf{positive charges} the field lines are directed radially outward.\\
		For \textbf{negative charges} the field lines are directed radially inward.\\
		The number of lines drawn leaving a charge is proportional to the \textbf{magnitude} of the charge.
	\section{Motion of a Charged Particle in a Uniform Field}
		Since the force is denoted as $\vec{F}_e = q\vec{E} = m\vec{a}$, then we can isolate $\vec{a}$ and determine the motion of the particle based on the acceleration. If the electric field is uniform, then we know the particle is under constant acceleration.
\chapter{Gauss's Law}
	\section{Electric Flux}
		The electric flux is defined as the number of total electric field lines multiplied by the area of the surface penetrated. We generalize the electric flux as:\\
		\centerline{$\Phi_E = EA \cos \theta$}
		This formula can be generalized as \\
		\centerline{$\Phi_E = \int_{}^{} \vec{E} \cdot d\vec{A}$}
		The direction of the area vector is chosen so that the vector points outward from the surface. If the electric field line points the same side as the area vector, then the electric flux through that area is positive, otherwise, it's negative.
	\section{Gauss's Law}
		The net flux through \textit{any} closed surface surrounding a point charge $q$ is given by $q/\epsilon_0$ and is independent of the shape of that surface. We define the net flux as:\\
		\centerline{$\Phi_E = \frac{q_{in}}{\epsilon_0}$}
		This implies that the net electric flux is the same through all closed surfaces. If the charge was outside the surface, the net flux due to that charge would be 0.
	\section{Application of Gauss's Law to Various Charge Distribution}
		Example: When calculating the electric field inside an \textbf{insulating }sphere, we define the internal charge by its density times its charge. We then replace it in $E = \frac{q_{in}}{\epsilon_0}$ to obtain\\
		\centerline{$E = k_e \frac{Q}{a^3}r$ for $r<a$}
		\\
		Example: When calculating the electric field from an infinite line of uniform charge, we define $q_{in}=\lambda l$ where $\lambda$ is the linear charge density. By then measuring the flux of a closed cylinder around the line, we define its area as $2\pi rl$ and we can then isolate the electric field as:\\
		\centerline{$E = 2k_e \frac{\lambda}{r}$}
		This would not be the case if the line was not infinitely long.\\
		Example: When finding the electric field due to an infinite plane of charge, we define $q_{in} = \sigma A$. By then using Gauss's Law, we find that \\
		\centerline{$E = \frac{\sigma}{2\epsilon_0}$}
		\\
		Using Gauss's only works for charge distributions with sufficient symmetry.
	\section{Conductors in Electrostatic Equilibrium}
		When there is no net motion of charge within a conductor, the conductor is in \textbf{electrostatic equilibrium}. A conductor in electristatic equilibrium has the following properties:
		\begin{enumerate}
			\item $E = 0$ everywhere inside conductor
			\item If conductor is isolated and carries a charge, the charge resides on the surface.
			\item The electric field at a point outside the conductor is perpendicular to the surface of the conductor and has magnitude $\sigma/\epsilon_0$ where $\sigma$ is the surface charge density at that point.
			\item On irregularly shaped conductors, the surface charge density is greatest at locations where the radius of curvature of the surface is smallest
		\end{enumerate}
\chapter{Electric Potential}
	\section{Electric Potential and Potential Difference}
		Like in Mechanics, the difference in potential energy is the force times the displacement or:\\
		\centerline{$\Delta U = -q \int_{B}^{A} \vec{E} \cdot d\vec{s}$}
		A charge in an electric field has a potential $U$. Dividing the potential energy by $q$ gives a physical quantity defined as the \textbf{electric potential}:\\
		\centerline{$V = \frac{U}{q}$}
		We can then define the potential difference $\Delta V$ as:\\
		\centerline{$\Delta V \equiv \frac{\Delta U}{q} = -\int_{A}^{B} \vec{E} \cdot d\vec{s}$}\\
		It's worth nothing that potential energy only exists in a system of two or more charges. If we have the potential difference and want to check the work done with a given charge, we define work as:\\
		\centerline{$W = q\Delta V$}
		We can then define electric field as this: The \textbf{electric field} is a measure of the rate of change of the electric potential with respect to position.
	\section{Potential Difference in a Uniform Electric Field}
		Since the electric field is uniform, the potential difference can be defined as:\\
		\centerline{$\Delta V = -Ed\cos \theta$}
		where $\theta$ is the angle between the electric field line and the direction of the displacement.\\
		Also, all points in a plane perpendicular to a uniform electric field are at the same potential.
	\section{Electric Potential and Potential Energy due to Point Charges}
		The equations speak for themselves:\\
		\centerline{Potential Difference: $V_B - V_A = k_e q[\frac{1}{r_B} - \frac{1}{r_A}]$}
		\centerline{Total Energy: $U = ke(\frac{q_1q_2}{r_{12}} + \frac{q_1q_3}{r_{13}} + \frac{q_2q_3}{r_{23}})$}
	\section{Obtaining the Value of the Electric Field from the Electric Potential}
		If we know the value of the electric potential, then assuming the electric field has only one component $E_x$, \\
		\centerline{$E_x = -\frac{dV}{dx}$}
		This adds onto the previously mentioned fact that the potential difference across an equipotential surface is 0.
	\section{Electric Potential Due to Continuous Charge Distributions}
		We can express the electric potential at a point due to a continuous charge distribution as the sum of the potential due to each point:\\
		\centerline{$V = k_e \int_{}^{} \frac{dq}{r}$}
		This is similar to calculating the electric. However, since electric potential is not a vector, the components do not cancel out.
	\section{Electric Potential Due to a Charged Conductor}
		\subsection{Electric Potential due to a Charged Conductor}
			The electric field is always going to be perpendicular to the equipotential surfaces surrounding a charged conductor.\\
			The surface of any charged conductor in electrostatic equilibrium is an equipotential surface: every point on the surface of a charged conductor in equilibrium is at the same electric potential. Furthermore, since the electric potential field is zero inside the conductor, the electric potential is constant everywhere inside the conductor and equal to its value at the surface.
		\subsection{A Cavity within a Conductor}
			Due to the equipotential property of a conductor, a cavity surrounded by a conductor is a field free region as long as there are no charges inside the cavity such as a insulating charged object.
	\section{Applications of Electrostatics}
		Wtf is a Van de Graaff Generator?? When a charged conductor touches the inside of a hollow conductor, all of the charge goes to the hollow conductor. Van de Graaff decided to build an electrostatic generator that uses an insulating belt to transfer charge to a metal dome. This metal dome can have extremely high potential. A 1m radius sphere can have a maxmimum potential of $ 3 \times 10^6 V$. This is useful to create large potential differences. \\
		\\
		Wtf is a Electrostatic Precipitator???! It's a device that removes particulate matter from combustion gases, thereby reducing air pollution. The device has a wire with a negative electric potential relative to the walls to create an electric field towards the wire. The electric field becomes high enough to cause a corona discharge, causing electrons and negative ions to charge dirt during collisions which then become attracted by the outer walls that are positively charged. Das cool..
\chapter{Current and Resistance}
	\section{Electric Current}
		Let's say charges are moving through a cross section of surface area A. Current is defined as how many charge flows through this surface in terms of time. We define instantaneous current $I$ as:\\
		\centerline{$I \equiv \frac{dQ}{dt}$ where $1A = 1C/s$}
	\section{Resistance}
		We define the current density as\\
		\centerline{$J \equiv \frac{I}{A} = \sigma E$}
		where $\sigma$ is defined as conductivity.\\
		As previously defined, $\Delta V = El$. We can redefine the current density as \\
		\centerline{$J = \sigma \frac{\Delta V}{l}$}
		Using this and $\Delta V = El$, we find that \\
		\centerline{$J = \sigma \frac{\Delta V}{l}$}
		We can rearrange this to obtain \\
		\centerline{$\Delta V = \frac{l}{\sigma} J = (\frac{l}{\sigma A}) I = RI$}
		which we define as Ohm's law($1 \Omega \equiv 1 V/A$).
		We also define the resistivity as $\rho = \frac{1}{\sigma}$, meaning the resistivity is inversely proportional to the conductivity. We can sum this up as:\\
		\centerline{$R = \rho \frac{l}{A}$}
		where $\rho$ also depends on temperature:\\
		\centerline{$\rho = \rho_0 [1 + \alpha(T - T_0)]$} 
	\section{Electrical Power}
		The equations speak for themselves:\\
		\centerline{$\frac{dU}{dt} = \frac{d}{dt}(Q\Delta V) = \frac{dQ}{dt}\Delta V = I\Delta V = P$}
\chapter{Magnetic Fields}
	\section{Analysis Model: Particle in a Field(Magnetic)}
		\begin{itemize}
			\item The magnetic force is proportional to the charge $q$ of the particle
			\item The magnetic force on a negative charge is directed opposite to the force on a positive charge moving in the same direction
			\item The magnetic force is proportional to the magnitude of the magnetic filed vector \textbf{$\vec{B}$}
			\item The magnetic force is proportional to the speed $v$ of the particle
			\item If the velocity vector makes an angle $\theta$ with the magnetic filed, the magnitude of the magnetic force is proportional to $\sin \theta$
			\item When a charged particle moves \textit{parallel} to the magnetic field vector, the magnetic force on the charge is zero
			\item When a charged particle moves in a direction not parallel to the magnetic field vector, the magnetic force acts in a direction perpendicular to both \textbf{$\vec{v}$} and \textbf{$\vec{B}$}(use right hand rule)
		\end{itemize}
		We define the magnetic force as:\\
		\centerline{$\vec{F_B} = q\vec{v} \times \vec{B}$}
		Note: It's important to know how to use the right hand rule. Seriously. We can calculate a numerical value of the force with:\\
		\centerline{$F_B = |q| vB \sin \theta$}
		Differences to electric field:
		\begin{itemize}
			\item The electric force vector is along the direction of the electric field, whereas the magnetic force vector is perpendicular to the magnetic field
			\item The electric force acts on a chraged particle regardless of whether the particle is moving, whereas the magnetic force acts on a charged particle only when the particle is in motion
			\item The electric force does work in displacing a charged particle, whereas the magnetic force associated with a steady magnetic field does no work when a particle is displaced because the force is perpendicular to the displacement of its point of application
		\end{itemize}
\end{document}