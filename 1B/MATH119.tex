\documentclass[12pt]{report}
\usepackage[margin=2cm, left=5cm]{geometry}
\usepackage{amsfonts}
\usepackage{amsmath}
\usepackage{amssymb}

\title{MATH 119 - Calculus 2 for Engineering}
\author{Andy Zhang}
\date{Winter 2014}

\begin{document}
\maketitle
\chapter{Approximation Methods}
	\section{Linear Approximation}
		By using the slope at point $(a,b)$, we're estimating the value $f(x)$ where $x$ is near $a$:\\
		\centerline{$L(x) = y = f(a) + f'(a)(x-a)$}
	\section{Bisection Method}
		By taking advantage of the Intermediate Value Theorem, which states that if $f(a) < 0$ and $f(b) > 0$ and $f(x)$ is a continuous function, then there must exist a $c \in [a,b]$ such that $f(c) = 0$. \\
		The idea is to use 2 points, bisect the interval into 2 intervals and check which one contains the root i.e. which one has a positive and a negative output.
	\section{Newton's Method(Not on exam)}
		If we can't solve $f(x) = 0$, then we can replace $f(x)$ with a linear approximation $L(x)$ and solve equation $L(x) = 0$ instead. This can be illustrated as an iterative formula which is described as:\\
		\centerline{The Linear Approximation: $L_{x_n}(x) = f(x_0) + f'(x_0)(x-x_0)$}\\
		\centerline{The Iterative Formula: $x_{n+1} = x_n - \frac{f(x_n)}{f'(x_n)}$}\\
	\section{Fixed-Point Iteration(Not on exam)}
		\textbf{Theorem: Convergence of Fixed-Point Iteration} Suppose that $f(x)$ is defined for all $x \in \mathbb{R}$, that it is differentiable everywhere, and that its derivative is always bounded(so that there are no points with vertical tangents). If the equation $f(x) = x$ has a solution(i.e. if $f(x)$ has a fixed point), and \textbf{if $|f'(x)| < 1$ for all values of x within some interval containing the fixed point}, then the sequence generated by letting $x_{n+1} = f(x_n)$ will converge, with \textit{any} choice of $x_0$. 
\chapter{Polynomials}
	\section{Polynomial Interpolation(Not on exam)}
		\subsection{Problem}
			Suppose we are given $n+1$ points and we would like to find a smooth curve which passes through all of them. The simplest such curve which passes through all of them of degree $n$.
		\subsection{Steps}
			\begin{enumerate}
			\item Let's say there are $n+1$ points. We define the polynomial as\\
			\centerline{$f(x) = a_nx^n + a_{n-1}x^{n-1} + \dots + a_1x + a_0$}
			\item Plug in each point into this polynomial and obtain $n$ functions.
			\item Incrementally find the difference between $i$ and $i+1$, where $\delta y_i = y_{i+1} - y_i$. As a result, this reduces the amount of variables in the polynomial
			\item After isolating the first coefficient, it becomes simple to isolate each other coefficients using the derived equations.
			\item We can then plug these coefficients into the original polynomial and obtain a result.\\
			OR you can use a general formula:\\
			\centerline{$ y = y_0 + x\Delta y_0 + x(x-1)\frac{\delta^2 y_0}{2!} + \dots + x(x-1)(x-2)\dots (x-n+1) \frac{\Delta^n y_0}{n!}$}
			\end{enumerate}
			Side Note: If each node is equidistant, then there's a simpler generalization:\\
			\centerline{$y = y_0 + \frac{x-x_0}{h}\Delta y_0 + \frac{(x-x_0)(x-x_1)}{2!h^2}\Delta^2 y_0 + \dots + \frac{(x-x_0)(x-x_1)\dots (x-x_{n-1})}{n!h^n}\Delta^ny_0$}
	\section{Taylor Polynomials}
		\subsection{Idea}
			The idea behind this is to use the secant line in the Linear Approximation and using Polynomial Interpolation to derive a polynomial that is similar to a function's shape due to its derivatives. By reaching the $n^{th}$ derivative where $n$ gets bigger and bigger, the polynomial resembles more and more to the function.\\
			tl;dr: $P_{n,x_0}(x) = \sum_{k=0}^{n} \frac{f^{(x)}(x_0)}{k!}(x-x_0)^k$\\
			This becomes really handy when you need to do complicated stuff on functions such as integrals since polynomials are easy to integrate.
		\subsection{Remainder Theorem}
			Suppose that $f$ has $n+1$ derivatives at $x_0$. Then \\
			\centerline{$f(x) = \sum_{k=0}^{n} \frac{f^{(k)}(x_0)}{k!}(x-x_0)^k+R_n(x)$}\\
			where\\
			\centerline{$R_n(x) = \int_{x_0}^{x} \frac{(x-t)^n}{n!} f^{(n+1)}(t)dt$}\\
			By applying some mathemagic on this, we obtain Taylor's Inequality:\\
			The error in using the $n^{th}$-order polynomial $P_{n,x_0}(x)$ as an approximation to $f(x)$ satisfies the inequality \\
			\centerline{$|R_n(x)| \leq K \frac{|x-x_0|^{n+1}}{(n+1)!}$}\\
			where $|f^{n+1}(t) \leq K$ for all values of $t$ between $x_0$ and $x$
			This allows us to find the error margin of the Taylor Polynomial when estimating values. 
\end{document}